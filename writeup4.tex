%% LyX 2.1.3 created this file.  For more info, see http://www.lyx.org/.
%% Do not edit unless you really know what you are doing.
\documentclass[english]{article}
\usepackage[T1]{fontenc}
\usepackage[latin9]{inputenc}
\usepackage{geometry}
\geometry{verbose,tmargin=1in,bmargin=1in,lmargin=1in,rmargin=1in}
\usepackage{setspace}
\usepackage{babel}
\begin{document}
\begin{spacing}{0.5}

\title{Assignment 4}
\end{spacing}

\begin{spacing}{0}

\author{Cierra Shawe}
\end{spacing}

\begin{spacing}{0}

\date{CS344 - Summer 2015}
\end{spacing}
\maketitle
\begin{enumerate}
\item \noindent A design for your system, including where your implementation
differed from this design if applicable. 


First of all, everything went poof on Saturday night when trying to
upload to git, and after that I kind of just gave up out of frustration,
which wasn't the best idea, but instead I just started reading about
sockets and preparing for assignment 5, because sometimes things happen
and life moves on. I'm not the first person to lose code by stupid
mistakes, and by no means will be the last.


Anyway, onto the design.


The first task is generating the list of primes, which can be done
using the sieve. The bitmap will be itterated through by using a constant
that is equal to the size of an unsigned int, take into consideration
where the boundaries are, and then from there just itterate though
like an array, just using math (pos {*} SIZE\_COST). That way, in
order to do this in parallel, each thread or proccess can take a chunk
of memory, after the first pass has been made on the first chunk of
memory. This gives the other proccesses or threads numbers to work
with, and help eliminate any non-primes. After the threads reads through
and eliminates all numbers in the first block, they can keep reading
on the other blocks. Since this is being done in parallel, a later
on thread will have to check through more primes, theoretically taking
more time. However, a mutex can be used for the code executing the
next block, so that any thread running on a part further in the program
has to make sure the previous block is fully accounted for, that way
it doesn't skip numbers. So if chunk one ran the first 1024 numbers,
all of the threads could run through that, but then only thread two
would have to finish before thread three can check it's numbers, then
all of the threads can run up to chunk 2, and once thread three finishes
it's part, then the other threads can use all of those, up until the
sqrt of MAX\_UINT, where at that point all of the threads are allowed
to work on their respective chunks of memory. The chunks will be an
arbituary number, based on the number of threads (aka divided up into
equal chunks). Then in the end, the result will be a map of either
a prime number, or 0.


To find a happy number, every number will end up producing either
a 0, 1, 4, 16,37, 43, 58, 89, 20, or 145. What I find interesting,
is that a sad number will end up looping through the numbers greater
than one indefinitely. If it ends up being anything but one, it can
be marked as a sad number, so it'll get marked as a 0 within the original
bitmap. Once everything is done running, it'll go through and count
how many happy numbers there are, then go through and create a second
bitmap to store everything in. Then, after it's done, move it over.
This way, the order is maintained, and all of the happy numbers will
be accounted for. 


Overall, my submission just includes something that can return all
of the happy numbers under a certain number. It's not that special,
it was literaly was just how far I got before giving up the second
time around. 

\item Your experimental results for the runtime of your implementations
with varying numbers of concurrent units.


N/A -> Because I lost my code, and there is no way I'll be able to
get everything working again in time, I'm unable to actually graph
this, however I'll just go over there results from before everything
went poof. 


What I noticed, that the threaded version was much faster than a single
process, and I would assume it would be faseter than the multi-proccess
version because it doesn't have to recreate the full program, it's
just the minimal ammount of instructions for the scheduler to handle
things. On my machine, which is a quad-core (never tested on osclass),
the threaded version was considerably faster up to about 16 threads,
then it felt like it didn't really make a difference. Given that I
was just timing this by how long I had to wait by glancing over at
my clock, it's not the best gauge, however the difference between
1 and 16 was huge, and then when I randomly tested 512, it ran about
the same as the 16 threads. 

\item \noindent A work log, detailing what you did when \textendash{} this
can fairly easily be created if you are using some form of revision
control. 

\begin{verbatim}
Tue 7/28 Theoretical Design, reading all about threads, and some on shared memory
Wed 7/29 Reading on shared memory, played some with example code, more design
(No internet for my computer from here, yay coding in the woods via car power!)
Thu 7/30 Got code to correctly find the happy numbers
Fri 7/31 Using mmap to store the giant array and sieve it out, then calc happy primes
Sat 8/1  Using threads to sieve and calc 
(Back home with internet (late saturday night))
commit 84017c lightnightninja <mywho09@gmail.com> 
Date: Sat Aug 1 19:26:25 2015 -0700
	I lost all of my code trying to create the last repository -.- 
	Everything.

commit 4009e2 lightnightninja <mywho09@gmail.com> 
Date: Sat Aug 1 20:29:16 2015 -0700
	So... Back to simple finding primes and happy numbers. 
	1/4 recovered(not even)... Woo... So Done... 

Sun 8/2 Too frustrated to continue, write a4 off as low grade.
		Started reading more about sockets. 
Mon 8/3 Finishing this writeup, submitting code, 
		moving on to a5 and grading.
\end{verbatim}
\selectlanguage{english}%
\item \noindent Any challenges you overcame in completing this assignment. 


\noindent I was having a hard time getting my happy number code to
work properly, there were always just little bugs within the code.
I kept running out of ram on my machine, which was always amusing.
So in order to combat that, I would just run smaller numbers, so that
the chance of having the contiguous space was higher. After that I
struggled with itterating through the memory map, but like in cs271,
we learned that a variable can be used in order to create a constant
offset to itterate by, which ended up making it easier, since it became
more like an array access.


\noindent Another challenge was being behind from the last assignment,
however having taken the time to get it working taught me so much
more about proccess creation, so I don't really regret that. On top
of that, it's just been stressful because the entire two weeks of
the assignment was choosing school work or spending time with people
I won't get to see for a really long time. That on top of still coming
off the the super messed up meds, and not being able to get an appointment
until today for something different has just left everything else
a mess. Still not an excuse, just what happened, and it's exactly
why my plan is to simply make up for it on the next assignment now
that everyone is gone, back on my much more stable meds, and the end
is drawing near. Can only move forward from here!

\item Answers to the following questions: 

\begin{enumerate}
\item What do you think the main point of this assignment is?


This assignment was supposed to teach us about the advantages of threading
and using shared memory, while also showing what can happen when things
aren't properly maintained, i.e things fight for the same memory space,
race conditions, etc. It also shows how threading, or even multiple
processes can be considerably faster than not doing things in paralell.
An interesting graph I saw showed the preformance gains and number
of processes, and when 95\% of the code is done in paralell using
threading, there can be almost a 20x reduction of time. Threading
also helps with utilizing the resources our computers have these days,
as we might as well be able to use all of the power that we have.
Mmap makes it so that there are simply fewer system calls, which reduces
the amount of time required to run, since file I/O is expensive (as
we learned in a2 I believe). 

\item How did you ensure your solution was correct?


To begin with I was just using arrays to make sure that the code was
able to find happy numbers properly. Then from there it was just outputting
the results, making it so that I could just read them off. I only
would compare with a list of happy primes from:


https://en.wikipedia.org/wiki/Happy\_number\#Happy\_primes then figured
it was good enough, and I had planned on doing more extensive checking
today. 

\item What did you learn?


ALWAYS CREATE YOUR GIT REPOSITORY BEFORE DOING ANYTHING -\_- ALWAYS.
Otherweise very bad things will happen if you don't do it correctly.
On top of that, I gained a better understanding how threading can
be used in order to speed up programs that are very computation heavy,
and that they can all share memory in order to make it more efficient.
I also learned that arrays can take up system memory very quickly,
and because that space might not be enough to hold the entire array,
things can go crash. It was entertaining watching my code take up
all 16 gigs of ram (some of this was probably virtual, no idea how
OSX handles it). Allocating space for every unsigned int takes up
a lot of memory space. 4{*}4294967296 is a lot of bytes (16GB). And
I always learning about new things with numbers. \end{enumerate}
\end{enumerate}

\end{document}
